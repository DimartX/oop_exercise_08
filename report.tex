\documentclass[12pt]{article}

\usepackage{fullpage}
\usepackage{multicol,multirow}
\usepackage{tabularx}
\usepackage{ulem}
\usepackage[utf8]{inputenc}
\usepackage[russian]{babel}
\usepackage{minted}

\usepackage{color} %% это для отображения цвета в коде
\usepackage{listings} %% собственно, это и есть пакет listings

\lstset{ %
language=C,                 % выбор языка для подсветки (здесь это С)
basicstyle=\small\sffamily, % размер и начертание шрифта для подсветки кода
numbers=left,               % где поставить нумерацию строк (слева\справа)
%numberstyle=\tiny,           % размер шрифта для номеров строк
stepnumber=1,                   % размер шага между двумя номерами строк
numbersep=5pt,                % как далеко отстоят номера строк от подсвечиваемого кода
backgroundcolor=\color{white}, % цвет фона подсветки - используем \usepackage{color}
showspaces=false,            % показывать или нет пробелы специальными отступами
showstringspaces=false,      % показывать или нет пробелы в строках
showtabs=false,             % показывать или нет табуляцию в строках
frame=single,              % рисовать рамку вокруг кода
tabsize=2,                 % размер табуляции по умолчанию равен 2 пробелам
captionpos=t,              % позиция заголовка вверху [t] или внизу [b] 
breaklines=true,           % автоматически переносить строки (да\нет)
breakatwhitespace=false, % переносить строки только если есть пробел
escapeinside={\%*}{*)}   % если нужно добавить комментарии в коде
}


\begin{document}
\begin{titlepage}
\begin{center}
\textbf{МИНИСТЕРСТВО ОБРАЗОВАНИЯ И НАУКИ РОССИЙСОЙ ФЕДЕРАЦИИ
\medskip
МОСКОВСКИЙ АВЦИАЦИОННЫЙ ИНСТИТУТ
(НАЦИОНАЛЬНЫЙ ИССЛЕДОВАТЬЕЛЬСКИЙ УНИВЕРСТИТЕТ)
\vfill\vfill
{\Huge ЛАБОРАТОРНАЯ РАБОТА №6} \\
по курсу объектно-ориентированное программирование
I семестр, 2019/20 уч. год}
\end{center}
\vfill

Студент \uline{\it {Артемьев Дмитрий Иванович, группа М8О-206Б-18}\hfill}

Преподаватель \uline{\it {Журавлёв Андрей Андреевич}\hfill}

\vfill
\end{titlepage}

\subsection*{Условие}

Задание №1: создать шаблонный класс коллекции стек, по заданию содержащей треугольники, с возможностью
\begin{enumerate}
\item push 
\item pop 
\item top 
\item insert 
\item erase 
\item count\_if
\item print
\end{enumerate}

Реализовать аллокатор, который выделяет фиксированный размер памяти. Внутри аллокатор должен хранить указатель на используемый блок памяти и динамическую коллекцию указателей на свободные блоки. Динамическая коллекция в моём варианте - динамический массив. 

\subsection*{Описание программы}

Исходный код лежит в 8 файлах:
\begin{enumerate}
\item src/main.cpp: основная интерактивная программа, возможность работать со стеком
\item include/stack.hpp: описание и реализация класса стека
\item include/figure.hpp: описание и реализация класса обобщённой фигуры
\item include/triangle.hpp: описание и реализация класса треугольника
\item include/template.hpp: разные шаблонные вещи
\item include/point.hpp: описание и реализация класса точки
\item include/allocator.hpp: описание и реализация класса аллокатора
\item include/vector.hpp: описание и реализация класса вектора
  
\end{enumerate}

\subsection*{Дневник отладки}

Я 6 долбанных часов дебажил маленькую описку в методе $erase$ класса vector!!!

\subsection*{Недочёты}

Двойная линия при вставке и удалении элемента по итератору. 

\subsection*{Выводы}

Я научился писать аллокаторы памяти для динамических коллекций, использовать библиотеку Boost Test.

\vfill

\subsection*{Исходный код}

{\Huge allocator.hpp}
\inputminted
    {C++}{include/allocator.hpp}
    \pagebreak

{\Huge vector.hpp}
\inputminted
    {C++}{include/vector.hpp}
    \pagebreak

{\Huge figure.hpp}
\inputminted
    {C++}{include/figure.hpp}
    \pagebreak

{\Huge point.hpp}
\inputminted
    {C++}{include/point.hpp}
    \pagebreak

{\Huge stack.hpp}
\inputminted
    {C++}{include/stack.hpp}
    \pagebreak

{\Huge template.hpp}
\inputminted
    {C++}{include/template.hpp}
    \pagebreak
    
{\Huge triangle.hpp}
\inputminted
    {C++}{include/triangle.hpp}
    \pagebreak
    
{\Huge main.cpp}
\inputminted
    {C++}{src/main.cpp}
    \pagebreak    
\end{document}
